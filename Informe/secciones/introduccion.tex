\section{Introducción}
El presente trabajo práctico aborda principalmente tres problemas:
\begin{enumerate}
  \item La implementación de una herramienta que permita realizar el \textit{traceroute} de un paquete
  desde su nodo origen hasta el destino, identificando todos los hops intermedios.
  \item La identificación de enlaces submarinos en las rutas de paquetes que tienen como destino
  nodos en otros continentes mediante el análisis estadístico de los datos obtenidos con el \textit{traceroute}
  \item Estimar el round trip time (RTT) y el throughput de una conexión simulándola mediante el envío de 
  sucesivos paquetes \textit{Echo Request} con una herramienta que denominaremos \textit{ping}
\end{enumerate}

Las dos herramientas mencionadas, \textit{traceroute} y \textit{ping}, fueron 
escritas en Python utilizan la biblioteca Scapy \footnote{Scapy: \url{http://www.secdev.org/projects/scapy/}} para la generación y comunicación 
de paquetes. Además, se basan en el envío de paquetes de tipo ICMP (Internet Control Message Protocol)\footnote{ICMP: \url{https://tools.ietf.org/html/rfc792}}.
 Se pueden utilizar distintos protocolos para implementar estas herramientas
pero actualmente predomina el uso de ICMP porque, al ser un protocolo de control,
los firewalls generalmente no los filtran.\par

La información que recolecta el \textit{traceroute} incluye el numero de hop \textit{(nodo utilizado en la ruta para llegar desde el host origen al destino)}, el RTT promedio de cada hop, y el \textit{z-score} o valor standard (ZRTT) de cada hop
con respecto al promedio de todos. Nos valemos del análisis de éstos datos para identificar los enlaces submarinos y para proponer una heurística que detecte los enlaces submarinos independientemente de la ruta utilizada.

La estimación más confiable del RTT hasta los destinos que formaron parte de nuestros experimentos la calculamos con la fórmula de Karn \& Patridge 
\footnote{Improving Round-Trip Time Estimates in Reliable Transport Protocols \url{http://ccr.sigcomm.org/archive/1995/jan95/ccr-9501-partridge87.pdf}}:

\begin{center}
  $SRTT_{i+1}$ = ($\alpha$ * $SRTT_i$) + (1 - $\alpha$) * $s_i$, donde $s_i$ es el i-esimo RTT estimado.
\end{center}

 El eventual throughput TCP de la conexión entre los host origen y destino lo calculamos, en los casos en los que había pérdida de paquetes con la fórmula de Mathis\footnote{\url{http://cseweb.ucsd.edu/classes/wi01/cse222/papers/mathis-tcpmodel-ccr97.pdf}}
\begin{center}
	throughput < $\frac{MSS}{RTT * \sqrt{p}}$
\end{center}

y, en los casos en que no había perdida de paquetes, utilizamos la formula para enlaces ideales:
\begin{displaymath}
  throughput = \frac{|Ventana|}{RTT}
\end{displaymath}
