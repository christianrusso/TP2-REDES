\section{Conclusiones}

La técnica utilizada para implementar el \textit{traceroute} permite identificar de una manera estadística la ruta que recorren la mayoría de los paquetes al \textbf{ir} hacia el host destino. Esto puede resultar útil para intentar identificar el país en el que se encuentra dicho host, aunque los servicios de geolocalización IP no son 100\% confiables ni precisos.


Utilizar simultáneos \textit{pings} para muestrear los RTT permite estimar de una manera \textit{fiel} el valor del RTT de la conexión contra un host determinado. Además, podemos adaptar la fórmula de Karn \& Patridge de la manera propuesta \textit{(calculando el valor de $\alpha$ a utilizar en función de la muestra)} para que el RTT estimado no sea afectado por picos en el RTT.

Que el valor del RTT se estime correctamente es de gran importancia ya que esto se utiliza para definir el timeout de retransmisión ha utilizar durante la transmisión y esto afecta directamente al throughput obtenido.


También pudimos comprobar empíricamente la importante relación que existe entre la probabilidad de pérdida de paquetes y el throughput de TCP, y lo mucho que el primero afecta al segundo.


Argentina se conecta mayoritiariamente con los países de Europa, Asia y Oceanía mediante enlaces con EEUU, y sólo con EEUU. No debiera ser así, dado que centralizar las conexiones allí puede generar un aumento de hops innecesario en la ruta, que aumente los RTT.


Por último, concluímos que el uso del valor estándar o valor Z del RTT (ZRTT) es suficiente para identificar enlaces submarinos \textit{(utilizando la heurística propuesta, como muy buenos resultados)}, ya que los cambios que se producen en este valor cuando se compara los dos extremos del enlace son fácilmente perceptibles.
