
\begin{figure}[ptb]
\includegraphics[scale=0.30]{logo.jpg}\hspace{6cm}
\includegraphics[scale=0.90]{logo_dc.jpg}
\end{figure}

%Datos de la caratula
\materia{Teoria de las Comunicaciones}
\titulo{Trabajo pr\'actico II}
\subtitulo{Rutas en Internet}
\hspace{6cm}
\integrante{Gomez, Fernando Nahuel}{695/11}{fernando.gmz12@gmail.com}
\integrante{Jabalera Gasperi, Fernando}{56/09}{fgasperijabalera@gmail.com}
\integrante{Russo, Christian Sebastián}{679/10}{christian.russo8@gmail.com}
\integrante{Tiffenberg, Valeria}{193/10}{valetiff@gmail.com}

\palabrasClave{Red, Scapy, IP, TCP, Internet, Protocolos de Red}
  % Reconocimiento caras. PCA. Power Method. Deflation. Autovalores. Autovectores. Matriz
  % semi definida positiva.

% \resumen{El presente trabajo analiza los algor\'itmos de resoluci\'on de sistemas de ecuaciones %
% Gauss y LU mediante una simulaci\'on del c\'alculo de Isotermas en hornos industriales. \\ %
% Expone un sistema de ecuaciones para detectar la Isoterma de 500 grados y analiza el
% comportamiento de cada algoritmo en funci\'on a la discretizaci\'on de los puntos dentro del
% horno, cantidad de instancias a resolver y variaci\'on de instantes.\\ % Finalmente, detalla las
% conclusiones obtenidas mediante la comparaci\'on de ambos algoritmos en las circunstancias
% mencionadas.\\
% }
\hypersetup{%
 % Para que el PDF se abra a página completa.
 pdfstartview= {FitH \hypercalcbp{\paperheight-\topmargin-1in-\headheight}},
 pdfauthor={Gomez, Gasperi, Russo, Tiffenberg},
 pdfsubject={TP1}
}

\parskip=5pt % 10pt es el tamaño de fuente

% Pongo en 0 la distancia extra entre ítemes.
\let\olditemize\itemize
\def\itemize{\olditemize\itemsep=0pt}

% Acomodo fancyhdr <- Creo que es el encabezado de pagina
\pagestyle{fancy}
\thispagestyle{fancy}
\addtolength{\headheight}{1pt}
\lhead{Gomez, Gasperi, Russo, Tiffenberg}
\rhead{1$^{do}$ Cuatrimestre 2015}
\cfoot{\thepage}
\renewcommand{\footrulewidth}{0.4pt}




%Pagina de titulo e indice
\thispagestyle{empty}

\maketitle
\tableofcontents

\newpage

